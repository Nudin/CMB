\documentclass[10pt,a4paper]{article}
%\documentclass[10pt,a4paper,twoside]{article}
\usepackage{a4wide}
\linespread{1.359140914229522617680} 
\usepackage[utf8]{inputenc}
\usepackage[T1]{fontenc}
\usepackage[german]{babel}

\author{Michael F. Schönitzer}
\title{Der Kosmische Mikrowellenhintergrund und seine Anisotropien}

\begin{document}
\maketitle

\section{Die Vorhersage und Entdeckung des CMB}
Albert Einstein und Willem de Sitter beschrieben 1917 zum ersten Mal das Universum als ganzes mit dem Formalismus der allgemeinen Relativitätstheorie unter der Annahme eines isotropen, homogenen und statischen Universums.
Nachdem Edwin Hubble 1925-29 entdeckte das alle Galaxien scheinbar von uns weg bewegen würden und diese Bewegung umso schneller ist je größer sie von uns entfernt sind, stellte Georges Lemaître fest, dass die Annahme das Universum sei statisch nicht notwendig richtig sei, sondern das Universum expandiert. Lemaître war auch der erste, der der Expansion des Universums darauf zurückführte, dass es dann in der Vergangenheit einen Zeitpunkt gegeben haben müsse, bei welchem das Universum Punktförming war -- die Urknall-Theorie war geboren.
Die Expansion kann durch die von Alexander Friedmann bereits 1922 aufgestellten Friedmann-Gleichungen beschrieben werden.
Der unterschied zwischen Einsteins und Friedmanns Modell, war dass Einstein die sogenannte Kosmologische Konstante in seinen Modell einführte und ihre Größe derart bestimmte, dass die Gleichungen ein statisches Universum beschrieben. Friedmann verwarf die unbegründete Annahme des statischen Universums und setze die Konstante gleich null.
Heute wissen wir da das Universum zwar nicht statisch ist sonder sich seit dem Urknall ausdehnt, die Kosmologische Konstante jedoch trotzdem nicht null ist.
Die physikalische Interpretation der kosmologischen Konstante entzieht sich bisher unserem Wissen, man bezeichnet sie heute oft verallgemeinert als Dunkle Energie.

Die allgemeine, aus der allgemeinen Relativitätstheorie folgende, Beschreibung des Universums hängt von einer Reihe von Parametern ab, welche entscheiden ob das Universum expandiert oder sich zusammenzieht und ob es flach oder gekrümmt ist. Diese zu bestimmen gehörte über lange Zeit zu den wichtigen Aufgaben der Kosmologie. Inzwischen kennen wir sie genau genug um sagen zu können, dass wir in einem flachen, seit dem Urknall expandierendem und dabei abkühlendem Universum leben.

In den 1940ern folgerten George Gamow, Ralph Alpher und Robert Herman aus der Urknalltheorie dass es einen Strahlungshintergrund in Mikrowellenbereich geben müsste:

Kurz nach dem Urknall entstanden im Universum eine Vielzahl an Elementarteilchen, welche mit der Strahlung im thermischen Gleichgewicht standen. Mit der Zeit sank durch die Expansion des Universums die Dichte und damit auch die Temperatur des Strahlungs-Materie-Gemischs immer weiter ab, bis nach etwa 380.000 Jahren die Temperatur auf etwa 3.000\ Kelvin gesunken war und somit die Protonen und Elektronen zu Wasserstoffatomen zusammengehen konnten.  % Sprachfuck, etc.
Man spricht hier man nicht ganz richtig von Rekombination. Nach dieser nahe zu schlagartigen Rekombination, waren gab es keine freien Elektronen mehr, wodurch die Photonen nicht mehr durch Thomson-Streuung abgelenkt wurden und von nun an frei durch das Universum laufen. Das Universum wurde sozusagen schlagartig Durchsichtig. Die Photonen von damals durchfliegen das Universum bis heute. Ihre Frequenzverteilung entspricht der eines Schwarzkörpers, da sie vor der Rekombination mit der Materie im thermischen Gleichgewicht lagen. Ihre Temperatur entsprach der des Gleichgewichts, wurde jedoch seither durch die kosmische Expansion stark rotverschoben. Die Temperatur der kosmischen Hintergrundstrahlung (englisch cosmic microwave background, kurz CMB) beträgt heute etwa 2,7 Kelvin.

Nach der erfolglosen Suche zahlreicher astrophysikalische Teams, entdeckten 1964 Arno Penzias und Robert Woodrow Wilson zufällig den Mirkrowellenhintergrund, als sie an einer neuen empfindlichen Antenne arbeiteten und ein Störsignal fanden, dass gleich war unabhängig davon in welche Richtung die Antenne ausgerichtet war. Nachdem das Signal auch nicht durch Reinigung der Antennen verschwand und sie keine Erklärung dafür fanden, wanden sich an andere Physiker und die Kosmologen um Robert Dicke in Princeton identifizierten das Signal als kosmische Hintergrundstrahlung. Seine Entdeckung gilt als eine der wichtigsten Entdeckungen der Kosmologie und als Bestätigung des Urknallmodels, untermauert wird dies dadurch, dass das Spektrum äußerst präzise einem Schwarzkörper-Spektrum entspricht und über den gesamten Himmel äußerst gleichmäßig ist, was zeigt das es sich dabei nicht um eine Überlagerung von vielen kleinen Stahlungsquellen handelt.

Die extreme Isotropie der Strahlung stellte jedoch auch ein Problem dar. Die heutigen Strukturen im Universum können sich nur gebildet haben, wenn auch im frühen Universum vor der Rekombination bereits Dichteschwankungen existiert haben. Diese müssten sich jedoch in Schwankungen in der Temperatur des CMB niedergeschlagen haben. Mann begann also direkt nach der Entdeckung der Hintergrundstahlung damit Anisotropien in ihr zu suchen. Von 1965 bis 1992, also fast drei Jahrzehnte blieb diese Suche erfolglos.\footnote{Mit Ausnahme einer Dipolabweichung, welche dadurch erklärt werden kann, das sich die Erde im Bezug auf den CMB bewegt. Dadurch kommt es zu einer Dopplerverschiebung von $10^{-3}$\%} %Wort Dipolabweichung
In dieser Zeit wurden sowohl durch theoretische Vorhersagen als auch durch die experimentellen Befunde (nämlich keine) die Vorhersagen der Stärke der Temperaturschwankungen von 10\% auf 0,001\%  herunter korrigiert. 
Als gerade ernste Zweifel an der Richtigkeit der Theorien aufkam, brachte der Satellit COBE den Durchbruch. Dieser konnte nicht nur bestätigen, dass das Spektrum extrem genau dem eines Schwarzkörpers entspricht, %dopplung
es fand auch die lange gesuchten Anisotropien.

Da man -- wie wir im folgenden sehen werden -- aus den Anisotropien des CMB viele wichtige Informationen gewinnen kann, insbesondere die oben angesprochenen Parameter der Beschreibung des Universums, wurden seitdem etliche Experimente unternommen, darunter auch zwei weitere Satellitenmissionen. Satelliten sind sind wie bei den meisten Wellenlängen auch bei Mikrowellenteleskopen von Vorteil, da die Atmosphäre Strahlung absorbiert und somit das Bild trübt. Die vielen anderen Experimente wurden entweder mit Ballonen in der Stratosphäre oder zumindest an kalten und/oder hochgelegenen Orten wie dem Südpol durchgeführt, da dort die Effekte der Atmosphäre schwächer sind.
Besonders erfolgreich waren die Ballonexperimente BOOMERanG und MAXIMA.
%mehr?
%WMAP, Plank, Daten <-> hier oder später?

\section{Die Ursachen der Anisotropien}


\end{document}