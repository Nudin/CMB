\documentclass[10pt,a4paper]{article}
%\documentclass[10pt,a4paper,twoside]{article}
\usepackage[utf8]{inputenc}
\usepackage[T1]{fontenc}
\usepackage[german]{babel}

\author{Michael F. Schönitzer}
\title{Der Kosmische Mikrowellenhintergrund und seine Anisotropien}

\begin{document}
\maketitle

\section{Die Vorhersage und Entdeckung des CMB}
Albert Einstein und Willem de Sitter beschrieben 1917 zum ersten Mal das Universum als ganzes mit dem Formalismus der allgemeinen Relativitätstheorie unter der Annahme eines isotropen, homogenen und statischen Universums.
Nachdem Edwin Hubble 1925-29 entdeckte das alle Galaxien scheinbar von uns weg bewegen würden und diese Bewegung umso schneller ist je größer sie von uns entfernt sind, stellte Georges Lemaître fest, dass die Annahme das Universum sei statisch nicht notwendig richtig sei, sondern das Universum expandiert. Lemaître war auch der erste, der der Expansion des Universums darauf zurückführte, dass es dann in der Vergangenheit einen Zeitpunkt gegeben haben müsse, bei welchem das Universum Punktförming war -- die Urknall-Theorie war geboren.
Die Expansion kann durch die von Alexander Friedmann bereits 1922 aufgestellten Friedmann-Gleichungen beschrieben werden.
Der unterschied zwischen Einsteins und Friedmanns Modell, war dass Einstein die sogenannte Kosmologische Konstante in seinen Modell einführte und ihre Größe derart bestimmte, dass die Gleichungen ein statisches Universum beschrieben. Friedmann verwarf die unbegründete Annahme des statischen Universums und setze die Konstante gleich null.
Heute wissen wir da das Universum zwar nicht statisch ist sonder sich seit dem Urknall ausdehnt, die Kosmologische Konstante jedoch trotzdem nicht null ist.
Die physikalische Interpretation der kosmologischen Konstante entzieht sich bisher unserem Wissen, man bezeichnet sie heute oft verallgemeinert als Dunkle Energie.

Die allgemeine, aus der allgemeinen Relativitätstheorie folgende, Beschreibung des Universums hängt von einer Reihe von Parametern ab, welche entscheiden ob das Universum expandiert oder sich zusammenzieht und ob es flach oder gekrümmt ist. Diese zu bestimmen gehörte über lange Zeit zu den wichtigen Aufgaben der Kosmologie. Inzwischen kennen wir sie genau genug um sagen zu können, dass wir in einem flachen, seit dem Urknall expandierendem und dabei abkühlendem Universum leben.

In den 1940ern folgerten George Gamow, Ralph Alpher und Robert Herman aus der Urknalltheorie dass es einen Strahlungshintergrund in Mikrowellenbereich geben müsste:

Kurz nach dem Urknall entstanden im Universum eine Vielzahl an Elementarteilchen, welche mit der Strahlung im thermischen Gleichgewicht standen. Mit der Zeit sank durch die Expansion des Universums die Dichte und damit auch die Temperatur des Strahlungs-Materie-Gemischs immer weiter ab, bis nach etwa 380.000 Jahren die Temperatur auf etwa 3.000\ Kelvin gesunken war und ab dann die Protonen und Elektronen zu Wasserstoffatomen zusammengehen konnten.  % Sprachfuck, etc.



\end{document}