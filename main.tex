\documentclass[10pt,a4paper]{article}
%\documentclass[10pt,a4paper,twoside]{article}
\usepackage{a4wide}
\linespread{1.359140914229522617680} 
\usepackage[utf8]{inputenc}
\usepackage[T1]{fontenc}
\usepackage[german]{babel}
\usepackage{amssymb}

\newcommand{\degree}{$^\circ$}

\author{Michael F. Schönitzer}
\title{Der Kosmische Mikrowellenhintergrund und seine Anisotropien}


\begin{document}
\maketitle

\section{Die Vorhersage und Entdeckung des CMB}
Albert Einstein und Willem de Sitter beschrieben 1917 zum ersten Mal das Universum als ganzes mit dem Formalismus der allgemeinen Relativitätstheorie unter der Annahme eines isotropen, homogenen und statischen Universums.
Nachdem Edwin Hubble 1925-29 entdeckte das alle Galaxien scheinbar von uns weg bewegen würden und diese Bewegung umso schneller ist je größer sie von uns entfernt sind, stellte Georges Lemaître fest, dass die Annahme das Universum sei statisch nicht notwendig richtig sei, sondern das Universum expandiert. Lemaître war auch der erste, der der Expansion des Universums darauf zurückführte, dass es dann in der Vergangenheit einen Zeitpunkt gegeben haben müsse, bei welchem das Universum Punktförming war -- die Urknall-Theorie war geboren.
Die Expansion kann durch die von Alexander Friedmann bereits 1922 aufgestellten Friedmann-Gleichungen beschrieben werden.
Der unterschied zwischen Einsteins und Friedmanns Modell, war dass Einstein die sogenannte Kosmologische Konstante in seinen Modell einführte und ihre Größe derart bestimmte, dass die Gleichungen ein statisches Universum beschrieben. Friedmann verwarf die unbegründete Annahme des statischen Universums und setze die Konstante gleich null.
Heute wissen wir da das Universum zwar nicht statisch ist sonder sich seit dem Urknall ausdehnt, die Kosmologische Konstante jedoch trotzdem nicht null ist.
Die physikalische Interpretation der kosmologischen Konstante entzieht sich bisher unserem Wissen, man bezeichnet sie heute oft verallgemeinert als Dunkle Energie.

Die allgemeine, aus der allgemeinen Relativitätstheorie folgende, Beschreibung des Universums hängt von einer Reihe von Parametern ab, welche entscheiden ob das Universum expandiert oder sich zusammenzieht und ob es flach oder gekrümmt ist. Diese zu bestimmen gehörte über lange Zeit zu den wichtigen Aufgaben der Kosmologie. Inzwischen kennen wir sie genau genug um sagen zu können, dass wir in einem flachen, seit dem Urknall expandierendem und dabei abkühlendem Universum leben.

In den 1940ern folgerten George Gamow, Ralph Alpher und Robert Herman aus der Urknalltheorie dass es einen Strahlungshintergrund in Mikrowellenbereich geben müsste:

Kurz nach dem Urknall entstanden im Universum eine Vielzahl an Elementarteilchen, welche mit der Strahlung im thermischen Gleichgewicht standen. Mit der Zeit sank durch die Expansion des Universums die Dichte und damit auch die Temperatur des Strahlungs-Materie-Gemischs immer weiter ab, bis nach etwa 380.000 Jahren die Temperatur auf etwa 3.000\ Kelvin gesunken war und somit die Protonen und Elektronen zu Wasserstoffatomen zusammengehen konnten.  % Sprachfuck, etc.
Man spricht hier man nicht ganz richtig von Rekombination. Nach dieser nahe zu schlagartigen Rekombination, waren gab es keine freien Elektronen mehr, wodurch die Photonen nicht mehr durch Thomson-Streuung abgelenkt wurden und von nun an frei durch das Universum laufen. Das Universum wurde sozusagen schlagartig Durchsichtig. Die Photonen von damals durchfliegen das Universum bis heute. Ihre Frequenzverteilung entspricht der eines Schwarzkörpers, da sie vor der Rekombination mit der Materie im thermischen Gleichgewicht lagen. Ihre Temperatur entsprach der des Gleichgewichts, wurde jedoch seither durch die kosmische Expansion stark rotverschoben ($z\approx 1.000$). Die Temperatur der kosmischen Hintergrundstrahlung (englisch cosmic microwave background, kurz CMB) beträgt heute etwa 2,7 Kelvin.

Nach der erfolglosen Suche zahlreicher astrophysikalische Teams, entdeckten 1964 Arno Penzias und Robert Woodrow Wilson zufällig den Mirkrowellenhintergrund, als sie an einer neuen empfindlichen Antenne arbeiteten und ein Störsignal fanden, dass gleich war unabhängig davon in welche Richtung die Antenne ausgerichtet war. Nachdem das Signal auch nicht durch Reinigung der Antennen verschwand und sie keine Erklärung dafür fanden, wanden sich an andere Physiker und die Kosmologen um Robert Dicke in Princeton identifizierten das Signal als kosmische Hintergrundstrahlung. Seine Entdeckung gilt als eine der wichtigsten Entdeckungen der Kosmologie und als Bestätigung des Urknallmodels, untermauert wird dies dadurch, dass das Spektrum äußerst präzise einem Schwarzkörper-Spektrum entspricht und über den gesamten Himmel äußerst gleichmäßig ist, was zeigt das es sich dabei nicht um eine Überlagerung von vielen kleinen Stahlungsquellen handelt.

Die extreme Isotropie der Strahlung stellte jedoch auch ein Problem dar. Die heutigen Strukturen im Universum können sich nur gebildet haben, wenn auch im frühen Universum vor der Rekombination bereits Dichteschwankungen existiert haben. Diese müssten sich jedoch in Schwankungen in der Temperatur des CMB niedergeschlagen haben. Mann begann also direkt nach der Entdeckung der Hintergrundstahlung damit Anisotropien in ihr zu suchen. Von 1965 bis 1992, also fast drei Jahrzehnte blieb diese Suche erfolglos.\footnote{Mit Ausnahme einer Dipolabweichung, welche dadurch erklärt werden kann, das sich die Erde im Bezug auf den CMB bewegt. Dadurch kommt es zu einer Dopplerverschiebung von $10^{-3}$\%} %Wort Dipolabweichung
In dieser Zeit wurden sowohl durch theoretische Vorhersagen als auch durch die experimentellen Befunde (nämlich keine) die Vorhersagen der Stärke der Temperaturschwankungen von 10\% auf 0,001\%  herunter korrigiert.
Die theoretische Grundlage dahinter ist, dass man erkannte, dass nicht die gewöhnliche sogenannte baryonische Materie sondern die Dunkle Materie im Universum vorherrscht.
Als gerade ernste Zweifel an der Richtigkeit der Theorien aufkam, brachte der Satellit COBE den Durchbruch. Dieser konnte nicht nur bestätigen, dass das Spektrum extrem genau dem eines Schwarzkörpers entspricht, %dopplung
es fand auch die lange gesuchten Anisotropien.

Da man -- wie wir in den beiden folgenden Kapiteln sehen werden -- aus den Anisotropien des CMB viele wichtige Informationen gewinnen kann, insbesondere die oben angesprochenen Parameter der Beschreibung des Universums, wurden seitdem etliche Experimente zu genaueren Vermessung der Schwankungen unternommen, wir werden sie in Kapitel \ref{Missionen} ansprechen.

\section{Die Ursachen der Anisotropien}
Bei den Ursachen der Schwankungen im CMB unterscheidet man zwischen primären und sekundären Anisotropien. Die primären Anisotropien sind die Effekte die zum Zeitpunkt der Entstehung des CMB entstanden, während die sekundären Anisotropien erst später auf dem Weg der Photonen durch das Universum entstanden.

Die wichtigsten Effekte der primären Anisotropien sind:
\begin{itemize}
\item Photonen aus Gebieten mit einer höheren Massendichte mussten aus dem Gravitationspotentialtopf entkommen, dabei verlieren sie einen Teil ihrer Energie, was einer Rotverschiebung entspricht. Photonen aus Gebieten niedrigerer Masse bekommen hingegen Energie vom Potential und werden ins blaue verschoben. Andererseits wird dieser Effekt dadurch teilweise kompensiert, dass die Gravitation zu einer Zeitdilatation führt. Daher stammen die Photonen der dichteren Regionen aus einer ein kleine bisschen früheren Zeit, zu welche das Universum noch heißer war. Beide Effekte werden zusammen als Sachs-Wolfe-Effekt beschrieben.
\item Die Dichteschwankungen im frühen Universum führen zu Geschwindigkeiten der Materie -- zusätzlich zur Geschwindigkeit der Expansion des Raumes (Pekuliargeschwindigkeit genannt). Die Elektronen mit denen die Photonen das letzte mal streuen haben also eine von der Dichte abhängige zusätzliche Geschwindigkeitskomponente. %mehr?
\item Wird in einem kleinem Gebiet die Baryonendichte erhöht, werden die Baryonen adiabatisch komprimiert und dadurch heißer. Das die Baryonen mit den Photonen im thermischen Gleichgewicht stehen werden somit auch die Photonen Energiereicher.
\end{itemize}

Zu den sekundären Anisotropien gehören insbesondere:
\begin{itemize}
\item Wir wissen heute das es zwischen dem Zeitpunkt der Rekombination und heute eine Reionisation gegeben haben muss. Daher gibt es bis heute freie Elektronen im Universum, an welchen die Photonen streuen können. Da die Thomson-Streuung weitgehend isotrop ist, ist die Richtung des Photons nach der Streuung weitgehend unabhängig von seiner Richtung vor der Streuung. Die gestreuten Photonen tragen keine Information über die Fluktuationen des CMB mehr. Dadurch werden die Anisotropien um den Faktor geschwächt, um den die Photonen eine Streuung erfahren.
\item Auf dem Weg zu uns durchlaufen die Photonen des CMB ein Universum in welchem sich die anfangs kleinen Dichteschankungen zu immer ausgeprägteren Strukturen entwickeln. Dabei durchlaufen sie eine viel zahl von Potentialtöpfen. Beim durchqueren eines Potenzialtopfes werden die Photonen analog zum Sachs-Wolfe-Effekt zunächst blau- und dann wieder rotverschoben werden. Das durchlaufen eines Potenzialtopfs mit gleich hohen Seiten verändert die Energie nicht. Über die kosmische Entfernungen gemittelt ist dies genau dann der Fall, wenn sich das Gravitationspotential im Universum insgesamt zeitlich konstant ist. Man kann zeigen das dies nur in einem Universum nach dem Einstein-de-Sitter-Modell der Fall ist -- in allen anderen kosmologischen Modellen kommt es zu einer Veränderung der Frequenzen der CMB-Photonen. Man nennt ihn den integrierten Sachs-Wolfe-Effekt.\cite{Schneider}
\item Beim durchlaufen des Gravitationspotentials der kosmischen Strukturen, werden die Photonen außerdem abgelenkt. Der Winkel unter welchem wir Photonen des CMB beobachten entspricht also nicht genau ihrer Position zum Zeitpunkt der Rekombination -- dadurch werden die Anisotropien auf kleinen Winkelskalen verschmiert.\cite{Schneider}
\item An den Elektronen des heißen Gases von Galaxienhaufen können Photonen streuen. %was zu richtung sagen? (schneeider 253)
Durch die Streuung ändert sich die Energie der Photonen ein wenig: sie haben nach der Compton-Streuung im Mittel eine höhere Frequenz. Dadurch wird die zahl der hochfrequenten Photonen relativ zum Plankspektrum erhöht, während die Zahl der niederfrequenten Photonen erniedrigt wird. Dies nennt man den Sunyaev-Zeldovich-Effekt, man kann ihn in den CMB-Daten erkennen, wenn man über ein breites Frequenzband misst. Man kann ihn damit einerseits aus den Daten herausrechnen, andererseits kann man ihn verwenden um Galaxiehaufen zu finden und ihre Gastemperatur und -dichte bestimmen.\cite{Schneider}
\end{itemize}

Darüber hinaus wird der kosmische Mikrowellenhintergund mit zahlreichen Quellen im Vordergrund überlagert. Diese werden manchmal als tertiäre Anisotropien bezeichnet.
%Beispiele
%rausrechnen

\section{Das Leistungsspektrum}
Die unterschiedlichen Ursachen der primären Anisotropie tragen auf unterschiedlich großen Skalen bei.

Hat man nun die Anisotropien des CMB gemessen und von Vordergrundquellen und sekundären Effekten bereinigt, betrachtet man die Stärke Anisotopien in Abhängigkeit ihrer räumlichen Größe.
Die klingt nach einer typischen Aufgabe für eine Fouriertransformation, da jedoch der von uns aus beobachtete CMB eine Kugeloberfläche darstellt, ist es nötig das Signal nicht in Trigometrische Funktionen, sondern Kugelflächenfunktionen zerlegen. Die mathematischen Grundlagen sind jedoch die selben. 
Man kann sich dies auch wie folgt vorstellen:
Man wählt zwei Punkte in festem Abstand $r$, misst ihre Temperatur und bestimmt ihre relative Temperaturdifferenz:
\begin{equation}
\Delta T = \frac{|T_1-T_2|}{\langle T\rangle }
\end{equation}
Wobei $\langle T\rangle$ die durchschnittliche Temperatur des CMB ist. Man berechnet dies für alle Punktpaare mit dem festen Abstand $r$ und mittelt über diese. Man erhält so den Wert des sogenanten Leistungspektrums $C(r)$ für den Wert $r$ und kann dies für ale Punktabstände wiederholen um das Leistungsspektrum als Funktion des Abstands $r$ zu erhalten. In der Praxis betrachtet man das Leistungsspektrum nicht in Abhängigkeit vom Punktabstand, sondern von der Ordnung $l$ der Kugelflächenfunktionen also $C_l$. $C_0$ stellt dabei die mittlere Temperatur dar, $T_1$ die stärke der dipolen Temperaturdifferenz, usw.
Dabei arbeitet man meist mit $l\left(l+1\right)C_l$, ein derartiges Leistungsspektrum ist in Abbildung \ref{pow} abgebildet.

Das aufgrund der obigen Effekte vorhergesagte und inzwischen auch gemessene Leistungsspektrum besteht dabei aus drei primären Teilen:
\begin{itemize}
\item Bei großen Skalen dominiert der Sachs-Wolfe-Effekt. Groß ist hierbei deutlich größer als die Horizontlänge zum Zeitpunkt der Rekombination. Die Horizontlänge gibt an, welches Distanz zwei Punkte maximal haben könnten, damit sie bei endlicher Lichtgeschwindigkeit $c$ seit dem Urknall je in einem kausalem Zugsamenhang bestehen konnten. Sie beträgt für ein flaches Universum etwa:
\[ \theta_{H,rec} \approx 1,8^\circ \]
\item Auf Winkelskalen $\ll\theta_{H,rec}$ betrachtet man Gebiete, welche vor der Rekombination innerhalb des Horizonts lagen, also mit einander gewechselwirkt haben. Hier kommt es zu den sogenannten Akustischen Schwingungen:
Das heiße Baryonen-Photonen-Gemisch besaß einen Druck, der bestrebt war der Schwerkraft entgegenzuwirken. Zog sich eine überdichte Materiewolke unter dem Einfluss der Schwerkraft zusammen, so presste der Druck des Baryonen-Photonen-Gemischs sie wieder auseinander, bis erneut die Schwerkraft überwog und sich das Gebiet wieder zusammenzog, und so fort. Durch das entgegenwirken von Schwerkraft und Druck kam es also zu Schwingungen im kosmischen Gemisch.
Zu derartigen Schwingungen kann es natürlich nur kommen, wenn die Materiewolke kleiner als die Horizontlänge ist. Konkret muss die Wolke klein genug sein, das die Druckwellen genügend Zeit haben um von einem Ende der Gaswolke zum anderen Ende zu kommen -- ansonsten baut sich der Druck nicht schnell genug auf um das Kollabieren aufzuhalten und die Wolke stürzt ohne zu Schwingen in sich zusammen. Das Gemisch aus Baryonen und Photonen lässt sich als eine relativistische Flüssigkeit betrachten und hat somit eine Schallgeschwindigkeit von $c_s\approx\sqrt{P/\rho}\approx c/\sqrt{3}$.\cite{Schneider} Man bezeichnet die daraus folgende maximale Ausdehnung als Schallhorizont, er unterscheidet sich aufgrund der hohen Schallgeschwindigkeit nur durch einen Faktor $1/\sqrt3$ von der oben definierten Horizontlänge.
Aufgrund des endlichen Alters der Universums und der Endlichkeit der Lichtgeschwindigkeit fangen die Wolken je nach Größe unterschiedliche früh an zu Schwingen, also sind die Schwingungen gleichgroßer Materiewolken synchronisiert. Ist eine Materiewolke also klein genug um zu schwingen, so schwingt sie phasengleich mit andern Wolken gleicher Größe. Zum Zeitpunkt der Rekombination sind also alle gleichgroßen Materiewolken in der gleichen Schwingungsphase. Dadurch könnten wir die Schwingungen deutlich im Leistungsspektrum sehen. Da es sich um Dichtewellen handelt, nennt man sie Akustischen Schwingungen; ein anderer physikalisch nicht ganz korrekter Name für die Peaks der Schwingungen ist Doppler-Peaks.
\item Solange die Rekombination nicht abgeschlossen ist, also noch elektrisch geladene Teilchen im kosmischen Gemisch enthalten sind, wechselwirken die Photonen mit dem kosmischen Gas. Dadurch treiben sie entstehende Materiewolke auseinander, sofern sie nicht massereich genug sind. Kleine Dichteschwankungen werden also weggedämpft, man bezeichnet diesen Effekt nach seinem Entdecker als Silk-Dämpfung (engl. Silk-Damping).\cite{A+R} Dazu kommt noch ein weiterer Effekt: Die Rekombination geschah nicht exakt instantan, sondern dauerte eine gewisse Zeit. Die Photonen des CMB kommen zu uns also aus einer Schale endliche Dicke. Betrachten wir nun einen Punkt im CMB, so beobachten wir die mittlere Temperatur über eine kleine Strecke entlang der Sichtlinie durch die Schale. Ist die Winkelskala auf der wir die Anisotropien beobachten kleiner als die Dicke der Schale, so befinden sich mehrere Maxima und Minimal innerhalb der Schalendicke, wodurch die Fluktuationen weggemittelt werden. Dieser Effekt wirkt auf den gleichen Skalen wie die Silk-Dämpfung. Durch diese Effekte werden auf Skalen $\lesssim 5'$ d.h. $l\gtrsim2500$ die Temperaturfluktuationen stark gedämpft und fallen rasch ab.\cite{Schneider}
\end{itemize}

\section{Kosmologische Parameter im CMB}
Die genaue Form des auf den besprochenen Effekten berechneten Leistungspektrums -- also die Stärke und Lage der Maxima und Minima ihre Abstände zueinander sowie die Stärke der Silk-Dämpfung und des Sachs-Wolfe-Effekts -- hängen teils äußerst empfindlich von den kosmologischen Parametern ab. Vermisst man also den Mikrowellenhintergrund äußerst präzise, so kann man aus dem Leistungsspektrum die kosmologischen Parameter äußerst präzise bestimmen.
%auslistung der Parameter - oder oben, oder unten?
Wir wollen dies an zwei Beispielen verdeutlichen:

Die Stärke und die Frequenz der akustischen Schwingungen wird durch die Stärke der Rückstellkraft, also des Drucks, bestimmt. Dieser hängt wiederum von der Zusammensetzung des kosmischen Gemisches ab.\cite{S+W00}

Die Größten Materiewolken die noch zu schwingen begonnen haben, hatten eine Ausdehnung die der Schallhorizontlänge entspricht. Daher definiert der Schallhorizont den Grundton des Mikrowellenhintergrunds, also die Lage des ersten Dopplerpeaks. Da wir den Zeitpunkt der Rekombination und die Schallgeschwindigkeit kennen, können wir die Schallhorizontlänge berechnen. Die berechnete Länge ist mit der Lage des ersten Maximums durch die Krümmung des Raumes verknüpft, denn eine bekannte Länge erscheint aus einer bekannten Entfernung unter einem unterschiedlich großem Winkel, je nachdem, ob und wie stark der Raum gekrümmt ist. In einem positiv gekrümmten (sphärischen) Raum erscheint er unter einem größerem Winkel, als in einem negativ gekrümmten (hyperbolischen) Raum. Durch exaktes Vermessen der Lage des ersten Maximums lässt sich also die Krümmung des Universums bestimmen. Diese ist außerdem direkt mit $\Omega_{tot}$ verknüpft.\cite{S+W03} %Omega_{tot} noch erklären

\subsection{Erforschung der Anisotropien}\label{Missionen}
Nach COBE wurden zahlreiche weitere Missionen gestartet um den CMB genauer zu vermessen, darunter auch zwei weitere Satelliten: WMAP und Plank.
Satelliten sind sind wie bei den meisten Wellenlängen auch bei Mikrowellenteleskopen von Vorteil, da die Atmosphäre Strahlung absorbiert und somit das Bild trübt. Darüber hinaus können sie den kompletten Himmel abmustern und somit 360\degree Ansicht aufnehmen können. Die Auflösung eines Teleskops hängt allgemein von der Öffnung des Teleskops und der Wellenlänge des zu detektierenden ab. Da die Größe eines Satelliten technisch stark begrenzt ist, können sie nicht die selbe Auflösung wie ein bodengestütztes Teleskop erreichen. Aufgrund der großen Wellenlänge der Hintergrundstrahlung ist die Auflösung von Mikrowellenteleskopen ohnehin relativ gering.

Die vielen anderen Experimente wurden entweder mit Ballonen in der Stratosphäre oder zumindest an kalten und/oder hochgelegenen Orten wie dem Südpol durchgeführt, da dort die Effekte der Atmosphäre schwächer sind. Sie haben den Vorteil einer vergleichsweise hohen Winkelauflösung.
%Besonders erfolgreich waren die Ballonexperimente BOOMERanG und MAXIMA.

Die 1989 gestattete Sonde COBE hatte eine Winkelauflösung von nur sieben Grad (420 Bogenminuten) und war somit nicht in der Lage die akustischen Schwingungen zu detektieren! Um die damals noch unsichere Existenz der akustischen Schwingungen zu detektieren und damit die kosmologischen Parameter zu bestimmen, wurden verschiedene erdgebundene Missionen gestartet. Den Ballon-Experimenten BOOMERanG und MAXIMA, sowie dem am Südpol stationiertem Projekt Dasi gelangen unabhängig von einander den ersten Dopplerpeak genau und den zweiten ansatzweise zu vermessen. Um die Genauigkeit der Vermessung des Leistungsspektrums zu erhöhen musste man nun erneut Satellitenmissionen starten, da Land- und Ballonexperimente durch ihr begrenztes Sichtfeld beziehungsweise durch ihre begrenzte Flugdauer nur einen kleinen Teil des Himmels aufnehmen können.
Der Satellit WMAP (früher: MAP oder auch Explorer 80) kartografierte von 2001 bis 2010 den CMB mit einer Winkelauflösung von etwa 15 Bodenminuten. 

\bibliography{lit}{}
\bibliographystyle{unsrtdin}

\end{document}
